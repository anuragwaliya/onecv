\documentclass[11pt]{moderncv}
\moderncvtheme[blue]{classic}
\usepackage[utf8]{inputenc}
\usepackage{color}

\firstname{Jeremy Kun}
\familyname{}
\title{Curriculum Vitae} 
\email{jkun2 /at/ uic.edu}

\definecolor{MidnightBlue}{rgb}{0.1, .35, .65}

\begin{document}
   \maketitle

      \section{Personal}
         \cvline{Name}
         {Jeremy Kun}

      \cvline{Thesis advisor}
         {Lev Reyzin}

      \cvline{Research summary}
         {I am a theoretical computer scientist with broad interests, including complexity theory, graph theory and network science, learning theory, cryptography, combinatorics, and geometry. My research to date focuses on theoretical and applied graph theory.}

      \cvline{Email}
         {jkun2 /at/ uic.edu}

      \cvline{Mailing Address}
         {Mathematics Department. University of Illinois at Chicago. 851 S Morgan St. Chicago, IL 60607-7045}

      \cvline{Webpage}
         {\href{http://math.uic.edu/~jkun2}{\textcolor{MidnightBlue}{\underline{\textbf{http://math.uic.edu/\textasciitilde{}jkun2}}}}}


   \section{Education}
         \cventry{2011 - Present}
         {University of Illinois at Chicago}
      {}
      {Ph.D in Mathematics.}
      {Expected 2016}
      {}

      \cventry{2007 - 2011}
         {California Polytechnic State University}
      {}
      {B.S. in Mathematics, Minor in Computer Science.}
      {Magna Cum Laude}
      {}

      \cventry{2011}
         {Budapest Semesters in Mathematics}
      {}
{}
      {Graduated with honors}
      {}


   \section{Publications}
         \cventry{2014}
         {\href{http://arxiv.org/abs/1401.3258}{\textcolor{MidnightBlue}{\underline{\textbf{A Boosting Approach to Learning Graph Representations}}}}}
      {Rajmonda Caceres, Kevin Carter, Jeremy Kun}
      {SIAM International Conference on Data Mining Workshop on Mining Networks and Graphs}
      {}
      {}

      \cventry{2014}
         {\href{http://arxiv.org/abs/1402.4376}{\textcolor{MidnightBlue}{\underline{\textbf{On Coloring Resilient Graphs}}}}}
      {Jeremy Kun, Lev Reyzin}
      {Mathematical Foundations of Computer Science}
      {}
      {}

      \cventry{2013}
         {\href{http://arxiv.org/abs/1308.3258}{\textcolor{MidnightBlue}{\underline{\textbf{Anti-Coordination Games and Stable Graph Colorings}}}}}
      {Jeremy Kun, Brian Powers, Lev Reyzin}
      {Syposium on Algorithmic Game Theory}
      {}
      {}


   \section{Preprints}
         \cventry{}
         {Approximation Lower Bounds and a new IP for Neighbor Aided Network Installation}
      {}
      {Alexander Gutfraind, Jeremy Kun, Ádám Lelkes, Lev Reyzin}
      {}
      {}

      \cventry{}
         {\href{http://arxiv.org/abs/1410.0245}{\textcolor{MidnightBlue}{\underline{\textbf{On the Computational Complexity of MapReduce}}}}}
      {}
      {Benjamin Fish, Jeremy Kun, Ádám Lelkes, Lev Reyzin, György Turán}
      {}
      {}

      \cventry{}
         {\href{http://arxiv.org/abs/1405.3210}{\textcolor{MidnightBlue}{\underline{\textbf{Locally Boosted Graph Aggregation for Community Detection}}}}}
      {}
      {Rajmonda Caceres, Kevin Carter, Jeremy Kun}
      {}
      {}


   \section{Talks}
         \cventry{2014}
         {On Resiliently Colorable Graphs}
      {Computer Science Seminar, University of Illinois at Chicago}
      {Research talk}
      {}
      {}

      \cventry{2014}
         {Resilient Coloring and Other Combinatorial Problems}
      {Midwest Theory Day. Purdue University}
      {Research talk}
      {}
      {}

      \cventry{2013}
         {Anti-Coordination Games and Stable Graph Colorings}
      {Computer Science Seminar, University of Illinois at Chicago}
      {Research talk}
      {}
      {}

      \cventry{2014}
         {How to Combine Graphs}
      {Chicago Area SIAM Student Conference, Northwestern University}
      {Graduate student talk}
      {}
      {}

      \cventry{2013}
         {Stable Graph Colorings, and Why You Should Care about NP}
      {Graduate Student Colloquium, University of Illinois at Chicago}
      {Graduate student talk}
      {}
      {}

      \cventry{2013}
         {A Brief Overview of Persistent Homology and its Applications}
      {Chicago Area SIAM Student Conference, University of Illinois at Chicago}
      {Graduate student talk}
      {}
      {}

      \cventry{2014}
         {Hybrid Images and Fourier Analysis}
      {Undergraduate Math Club, University of Illinois at Chicago}
      {Undergrad talk}
      {}
      {}

      \cventry{2014}
         {Elliptic Curves, Projective Geometry, and Python}
      {Stanford Pre-Collegiate Studies}
      {High school talk}
      {}
      {}

      \cventry{2013}
         {Classic Nintendo Games are NP-Hard}
      {Undergraduate Math Club, University of Illinois at Chicago}
      {Undergrad talk}
      {}
      {}

      \cventry{2012}
         {PageRank and the Billion-Dollar Eigenvector}
      {University of Illinois at Chicago Undergraduate Math Club}
      {Undergrad talk}
      {}
      {}

      \cventry{2011}
         {Eigenfaces: Linear Algebra for Facial Recognition}
      {University of Illinois at Chicago Undergraduate Math Club}
      {Undergrad talk}
      {}
      {}

      \cventry{2011 - Present}
         {Guest lectures to high school students}
      {Various locations}
      {High school talk}
      {}
      {}


   \section{Professional Programs}
         \cventry{June 2014}
         {Network Science Week}
      {American Mathematical Society Mathematics Research Community}
      {Received mentoring, engaged in research to attack open problems, and developed new collaborations}
      {}
      {}

      \cventry{Summer 2013}
         {Ph.D Student Intern}
      {MIT Lincoln Labs}
      {Research on machine learning in large graphs}
      {}
      {}

      \cventry{Summer 2012}
         {Ph.D Student Intern}
      {Lawrence Livermore National Laboratory}
      {Data mining research in wind energy and plasma physics}
      {}
      {}

      \cventry{Summer 2009}
         {Software Developer Intern}
      {Amazon.com}
      {Worked on the message-passing framework in a million-line service-oriented C++ architecture which regulated inventory in all of Amazon's warehouses}
      {}
      {}


   \section{Service}
         \cventry{2014}
         {Publicity Co-Chair}
      {ISAIM 2014}
      {International Symposium on Artificial Intelligence and Mathematics}
      {}
      {}

      \cventry{2013 - Present}
         {Organizer}
      {Graduate Student Theoretical Computer Science Seminar}
      {University of Illinois at Chicago}
      {}
      {}

      \cventry{2013}
         {Instructor}
      {Website Workshop}
      {Association for Women in Mathematics, University of Illinois at Chicago}
      {}
      {}


   \section{Awards}
         \cventry{2014}
         {Dean's Scholar Award}
{}
      {Granted by University of Illinois at Chicago}
      {}
      {}

      \cventry{2011}
         {Charles J. Hanks Excellence in Mathematics Award}
{}
      {Granted by California Polytechnic State University}
      {}
      {}

      \cventry{2010}
         {Robert P. Balles Mathematics Award}
{}
      {Granted by California Polytechnic State University}
      {}
      {}

      \cventry{2007}
         {Eagle Scout Award}
{}
      {Granted by Boy Scouts of America}
      {}
      {}

      \cventry{2009}
         {3rd Place in a Collegiate Regional Programming Contest}
{}
      {Granted by Association for Computing Machinery}
      {}
      {}


   \section{Teaching}
         \cventry{Intro Comp Sci}
         {TA}
      {University of Illinois at Chicago}
      {Spring 2012, Fall 2012, Spring 2013}
      {}
      {Wrote a grading robot for all labs and projects}

      \cventry{Calculus 1}
         {TA}
      {University of Illinois at Chicago}
      {Fall 2011, Fall 2013}
      {}
{}


   \section{Other}
         \cventry{Blog}
         {\href{http://jeremykun.com}{\textcolor{MidnightBlue}{\underline{\textbf{Math Intersect Programming}}}}}
      {}
      {In-depth presentation of technical topics, with full implementations in code. As of September 2014: 189 published posts, 2000 word average post length, 1.7 million page views since June 2011}
      {}
      {}



\end{document}
